\section{Light field geometry and light transport}

We consider 4D light fields -- two spatial dimensions and two angular
dimensions -- using the two-plane convention for the angular dimensions:
$\LF\GP{s, t, u, v}$ represents the intensity of light hitting the 
light field's plane at position $\GP{s,t}$ and along the angle indicated
by $\GP{u,v}$.  We discretize each light field into $N$ pixels and $K$ 
angles using a basis expansion:
\begin{align}
    \label{eqn,basis}
    \LF\GP{s,t,u,v; \bl}
    &=
    \sum_{k=1}^K b^k_u\GP{s, u} b^k_v\GP{t, v} 
    \sum_{i=1}^N b^i_s\GP{s} b^i_t\GP{t}
    \ell_{ki},
\end{align}
with spatial basis functions $b^i_s$ and $b^i_t$, angular basis 
functions $b^k_u$ and $b^k_v$, and expansion coefficients 
$\bl = \GB{\ell_{ki}}$.  Following convention, we use rectangular
pixels for the spatial basis functions, \eg{}
\begin{align}
    b^i_s\GP{s}
    &=
    b\GP{\frac{s - s_i}{\Delta_s}};
    \nonumber \\
    b\GP{x}
    &=
    \begin{cases}
        1, & \GM{x} < \frac{1}{2}; \\
        0, & \text{else.}
    \end{cases}
    \label{eqn,box}
\end{align}
To simplify the computational process of transporting a light field from
one plane to another, through masks and lenses, we discretize the angular
coordinates such that each angle (corresponding to a single value of $k$)
can be transported independently.

\subsection{Angular discretization and the optical plane}

To make this work, we designate a special plane the {\it angular plane} of
the discretization.  The angular coordinate of light fields in the system are
discretized according either {\it where} or {\it at what angle} they cross
the angular plane.  A common choice for modelling light transport inside a 
camera is to place the angular plane along the plane of the main lens, but we
will use other discretizations when modeling light fields external to the
camera.

In this work, we consider the case where the angular plane is discretized along
a lattice of $K$ points, however, our approach can be extended to other
discretizations. Let $\GC{\GP{u_k, v_k}}$ be $K$ locations on the angular plane
drawn from a lattice with sampling rate $\Delta_u$ in the $s/u$ direction and
$\Delta_v$ in the $t/v$ direction.  Finally, let $\bR$ be the affine coordinate
transformation from the light field $\LF$ to the angular plane.  

An common choice uses Dirac $\delta$ functions and discretizes the light field
by the location at which rays hit the optical plane, \eg
\begin{align}
    b^k_u\GP{s, u}
    &=
    \delta\GP{\frac{\bR_s\GP{s,u} - u_k}{\Delta_u}}.
\end{align}
If the angular plane is placed at the main lens, this approach essentially
represents the camera as a superposition of $K$ pinhole cameras, with the
pinholes apertures placed at the locations $\GC{\GP{u_k, v_k}}$.  A more
physically accurate model uses rectangular regions of the angular plane, \eg
\begin{align}
    b^k_u\GP{s, u}
    &=
    b\GP{\frac{\bR_s\GP{s,u} - u_k}{\Delta_u}},
\end{align}
using the box function~\eref{eqn,box}.  With the angular plane on the main
lens, this models the camera as a superposition of cameras with rectangular
apertures on the main lens.  In both the pinhole and rectangular aperture
cases, finer discretizations of the main lens/optical plane yield better
approximations of the often nonseparable main lens blur.

When modelling the light field of an object, it may be more natural to discretize
angle in a way that is intrinsic to the object rather than the camera.  This
may make it easier to use the compose the object's light field in \eg VR 
applications.  In these cases, we place the optical plane along a face of the
object's bounding box and discretize by the {\it angle} at which rays pass
through that plane instead of the location, \eg
\begin{align}
    b^k_u\GP{s,u}
    &=
    \delta\GP{\frac{\bR_u\GP{s,u} - u_k}{\Delta_u}}.
\end{align}
This representation is similar to the angular discretization of the Radon
transform.  Naturally, one can use either the Dirac $\delta$ or box function
(or any other orthogonal family) for the basis function.

\subsection{Light field linear algebra}

Let $\GP{\bl^p, \bR^p}$ and $\GP{\bl^q, \bR^q}$ be the coefficients of two
light fields using the same angular discretization and the coordinate
transforms from each light field to the optical plane.  Assume that light
from the plane $p$ is transported to the plane $q$; \ie{} $p$ is closer to the
scene and $q$ is closer to the detector.  We assume that the basis coefficients
on plane $p$, $\bl^p$, are known and we want to find the coefficients of the
light field on the plane $q$, $\bl^q$.  To accomplish this, we solve the 
approximation problem using the $L_2$ norm:
\begin{align}
    \bl^q 
    &=
    \argmin_{\bl}
    \GN{%
        \LF^q\GP{s,t,u,v; \bl}
        -
        \LF^p\GP{\bR^{pq}\GP{s,t,u,v}; \bl^p}
    },
    \label{eqn,xport}
\end{align}
where $\bR^{pq} = \GP{\bR^p}^{-1}\GP{\bR^q}$ is the affine coordinate
transformation from the plane $q$ to the plane $p$.  Due to the basis 
expansion~\eref{eqn,basis}, $\LF^q$ and $\LF^p$ lie on a finite-dimensional
subspace of $L_2$ and the solution to~\eref{eqn,xport} can be expressed as a 
linear operation on the coefficients $\bl^p$:
\begin{align}
    \bl^q
    &=
    \GB{\bD^q}^{-1} \bB^{qp} \bl^p,
\end{align}
and the entries of $\bD$ and $\bB^{pq}$ come from the inner products on
$L_2$.  Due to the $(s,u)/(t,v)$ separability of the 
expansion~\eref{eqn,basis}, these expressions are also separable in the
$(s,u)/(t,v)$ dimensions:
\begin{align}
    \GB{\bD^q}_{ik, ik}
    &=
    \GN{%
        b^k_u\GP{s,u} b^i_s\GP{s}
    }^2
    \times
    \GN{%
        b^k_v\GP{t,v} b^i_t\GP{t}
    }^2,
    \nonumber \\
    \GB{\bB^{qp}}_{ik_1, jk_2}
    &=
    \GA{%
        b^i_s\GP{s} b^{k_1}_u\GP{s,u}, 
        b^j_s\GP{\bR^{pq}_{su}\GP{s,u}} b^{k_2}_u\GP{\bR^{pq}_{su}\GP{s,u}}
    }
    \times
    \GA{%
        b^i_t\GP{t} b^{k_1}_v\GP{t,v},
        b^j_t\GP{\bR^{pq}_{tv}\GP{t,v}} b^{k_2}_v\GP{\bR^{pq}_{tv}\GP{t,v}} 
    }.
    \label{eqn,iprod}
\end{align}
Note that the angular functions $b^{k_1}_u \circ \bR^{pq}_{su}$ and $b^{k_2}_u$
are governed by the same discretization on the optical plane:
\begin{align}
    b^{k}_u\GP{\bR^{pq}_{su}\GP{s,u}}
    &=
    f_u\GP{\frac{\bR^p_{su*}\GP{\bR^{pq}_{su}\GP{s,u}} - u_{k}}{\Delta_u}} 
    \nonumber \\
    &=
    f_u\GP{\frac{\bR^{q}_{su*}\GP{s,u} - u_{k}}{\Delta_u}} 
    \nonumber \\
    &=
    b^{k}_u\GP{s,u},
    \label{eqn,same,angle}
\end{align}
where $f_u$ is the chosen angular basis function (typically either 
$\delta$ or the box function $b$) and $\bR_{su*}$ extracts either the
spatial ($s$) or angular ($u$) coordinate, depending on the optical
plane's discretization.  Consequently, assuming the discretization of the
optical plane is nonoverlapping,
\begin{align}
    k_1 \ne k_2 \Rightarrow \bB^{qp}_{ik_1, jk_2} = 0,
    \label{eqn,xport,block}
\end{align}
and $\bB^{qp}$ is block-diagonal with $K$ blocks: $\bB^{qp} = \diag_{k}
\bB^{qpk}$.  Practically, this means that the $k$th light field angle can be
transported through the system independently of the other $K-1$ angles. If each
of the light fields is high resolution ($N$ large), or if the angular plane
discretization is fine ($K$ large), this can lead to significant memory savings.

\subsection{Explicit transport expressions}

In this section, we provide expressions for 

\subsection{Light field rebinning}

So far we have considered light transport between two light fields sharing the
same optical plane and discretization.  However, it is sometimes helpful to
relate light fields with different angular discretizations, \eg to simulate a
different lens aperture or to combine light fields from two different cameras.
In this section we consider light transport between two light fields {\it on
the same plane} in space but {\it with different angular discretizations}.
Although we lose the convenient block structure from light transport using the
same discretization~\eref{eqn,xport,block}, the resulting expressions are 
still computationally efficient.

